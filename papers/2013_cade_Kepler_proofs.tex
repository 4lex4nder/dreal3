\documentclass[envcountsect]{llncs}
%\usepackage{stmaryrd,amsmath,amssymb,newlfont,graphicx,caption,verbatim}
\usepackage{amsmath,amssymb,newlfont,graphicx,caption,verbatim}
\usepackage[ruled,lined,boxed,commentsnumbered,linesnumbered]{algorithm2e}
\usepackage{mathpartir}
%\usepackage{hyperref}

\newcommand{\Var}{\mathop{\mathit{Var}}}
\newcommand{\Env}{\mathop{\mathit{Env}}}
\newcommand{\dom}{\mathrm{dom}}
\newtheorem{notation}[theorem]{Notation}
\newcommand{\len}{\mathit{len}}
\newcommand{\poly}{\mathsf{poly}}

\setlength{\textwidth}{5.3in}
%\setlength{\textheight}{8.2in}
%\setlength{\topmargin}{0in}
\setlength{\oddsidemargin}{.6in}
\setlength{\evensidemargin}{.6in}


\title{Extracting Proofs from a $\delta$-Complete Decision Procedure}
\author{Sicun Gao \and Soonho Kong \and Edmund M. Clarke}
\institute{Carnegie Mellon University, Pittsburgh, PA 15213}

\begin{document}
\maketitle

\begin{abstract}
\end{abstract}

\section{Introduction}

\section{$\delta$-Complete Decision Procedures}

\section{Interval Constraint Propagation}



\section{A Proof System over Intervals}

\begin{definition}[Language]
We work under the first-order theory with the following signature 
$$\mathcal{L}_{I} = \langle \emptyset, \cap, \cup, \subseteq,
\mathcal{F}\rangle$$
\end{definition}

\begin{notation}
To emphasize that we work over intervals, we write the variables using capital
letters $A, B, C, ...$
\end{notation}


\begin{definition}[Semantics]

\end{definition}



\begin{definition}[Axioms]
 
\end{definition}





Real interval $\mathbb{IR} = \{ [a, b] \in \mathbb{R} \times
\mathbb{R} \mid a \le b \}$ denotes a set of closed ranges of real
numbers. Let $\Var$ be the set of all variables. An environment $e \in
\Env : \Var \to \mathbb{IR}$ maps a variable $x \in \Var$ to a
corresponding real interval $e[x] \in \mathbb{IR}$. Interval
abstraction function $\alpha : 2^{\mathbb{R}} \to \mathbb{IR}$ takes a
set of real numbers $S$ and returns a minimal interval which covers
the given set:
\[
\alpha(S) = [\min(S), \max(S)].
\]
Similarly, we define concretization function $\gamma : \mathbb{IR} \to
2^{\mathbb{R}}$:
\[
\alpha([a, b]) = \{ x \mid a \le x \le b \},
\]
and join operator $\sqcup : \mathbb{IR} \times \mathbb{IR} \to
\mathbb{IR}$:
\[
[a,b] \sqcup [c, d] = [\min(a,c), \max(b, d)].
\]


For a function $f : \mathbb{R}^n \to \mathbb{R}$, we define its
interval extension $f^I : \mathbb{IR}^n \to \mathbb{IR}$ as
\[
f^I ([l_1, u_1], \dots, [l_n, u_n]) =
\alpha(\{ f(x_1, \dots, x_n) \mid x_i  \in [l_i, u_i] \}).
\]
Let $\mathcal{F}$ be a finite subset of real functions such that
$\mathcal{F} \subseteq \{f_i \mid f_i : \mathbb{R}^n \to \mathbb{R}
\}$. We define the interval extension $\mathcal{F}^I$ of $\mathcal{F}$
by taking interval extension of each real function $f \in
\mathcal{F}$:
\[
\mathcal{F}^I = \{ f^I \mid f \in \mathcal{F} \}.
\]

\paragraph{Over-Approximated  function}

A function $\widehat{f^I} : \mathbb{IR}^n \to \mathbb{IR}$ is an
over-approximated function for an interval extension of function $f$ if
$f^I$ satisfies the following condition:
\[
\forall I \in \mathbb{IR}^n. \
f^I(I) \subseteq \widehat{f^I}(I)
\]


For a given set of constraints $\mathcal{F}$, our SMT solver
generates a proof if the result is unsatisfiable. The proof system is
given as follows

\paragraph{Branch}
\begin{mathpar}
  \inferrule{
    A \cap C_1 \subseteq \emptyset \\ \cdots \\
    A \cap C_n \subseteq \emptyset \\
    B \subseteq C_1 \cup \cdots \cup C_n
  }{
    A \cap B \subseteq \emptyset
  }
\end{mathpar}

\paragraph{Over-Approximated Interval Arithmetic}
\begin{mathpar}
  % ADD
  \inferrule{
    f^I_1[a, b] \subseteq [a_1, b_1]
    \and
    f^I_2[a, b] \subseteq [a_2, b_2]
  }
  {
    (f_1 + f_2)^I[a, b] \subseteq [a_1 + a_2, b_1 + b_2]
  }
  \and
  % SUBTRACT
  \inferrule{
    f^I_1[a, b] \subseteq [a_1, b_1]
    \and
    f^I_2[a, b] \subseteq [a_2, b_2]
  }
  {
    (f_1 - f_2)^I[a, b] \subseteq [a_1 - b_2, b_1 - a_2]
  }
  \and
  % MULTIPLY
  \inferrule{
    f^I_1[a, b] \subseteq [a_1, b_1]
    \and
    f^I_2[a, b] \subseteq [a_2, b_2]
  }
  {
    (f_1 \times f_2)^I[a, b] \subseteq
    [
    \min (a_1 \times a_2, a_1 \times b_2, b_1 \times a_2, b_1 \times b_2),
    \max (a_1 \times a_2, a_1 \times b_2, b_1 \times a_2, b_1 \times b_2)
    ]
  }
  \and
  % DIVIDE
  \inferrule{
    f^I_1[a, b] \subseteq [a_1, b_1]
    \and
    f^I_2[a, b] \subseteq [a_2, b_2]
    \and
    0 \not \in [a_2, b_2]
  }
  {
    (f_1 / f_2)^I[a, b] \subseteq
    [
    \min (a_1 / a_2, a_1 / b_2, b_1 / a_2, b_1 / b_2),
    \max (a_1 / a_2, a_1 / b_2, b_1 / a_2, b_1 / b_2)
    ]
  }
  % SQRT
  \and
  \inferrule{
    f^I [a, b] \subseteq [a', b']
    \and
    a' \ge 0
  }
  {
    (\sqrt f)^I [a, b] \subseteq
    [\sqrt a', \sqrt b']
  }
  % LOG
  \and
  \inferrule{
    f^I [a, b] \subseteq [a', b']
    \and
    a' > 0
  }
  {
    (\log f)^I [a, b] \subseteq
    [\log a', \log b']
  }
  % EXP
  \and
  \inferrule{
    f^I [a, b] \subseteq [a', b']
  }
  {
    (\exp f)^I [a, b] \subseteq
    [\exp a', \exp b']
  }
  % SIN
  \and
  \inferrule{
    f^I [a, b] \subseteq [a', b'] \and
    \widehat{\sin} [a', b'] = [a'', b'']
  }
  {
    (\sin f)^I [a, b] \subseteq
    [a'', b'']
  }
  % COS
  \and
  \inferrule{
    f^I [a, b] \subseteq [a', b'] \and
    \widehat{\cos} [a', b'] = [a'', b'']
  }
  {
    (\cos f)^I [a, b] \subseteq
    [a'', b'']
  }
  % TAN
  \and
  \inferrule{
    f^I [a, b] \subseteq [a', b'] \and
    \widehat{\tan} [a', b'] = [a'', b'']
  }
  {
    (\tan f)^I [a, b] \subseteq
    [a'', b'']
  }
  % ARCSIN
  \and
  \inferrule{
    f^I [a, b] \subseteq [a', b'] \and
    \widehat{\arcsin} [a', b'] = [a'', b'']
  }
  {
    (\arcsin f)^I [a, b] \subseteq
    [a'', b'']
  }
  % ARCCOS
  \and
  \inferrule{
    f^I [a, b] \subseteq [a', b'] \and
    \widehat{\arccos} [a', b'] = [a'', b'']
  }
  {
    (\arccos f)^I [a, b] \subseteq
    [a'', b'']
  }
  % ARCTAN
  \and
  \inferrule{
    f^I [a, b] \subseteq [a', b'] \and
    \widehat{\arctan} [a', b'] = [a'', b'']
  }
  {
    (\arctan f)^I [a, b] \subseteq
    [a'', b'']
  }
  % SINH
  \and
  \inferrule{
    f^I [a, b] \subseteq [a', b'] \and
    \widehat{\sinh} [a', b'] = [a'', b'']
  }
  {
    (\sinh f)^I [a, b] \subseteq
    [a'', b'']
  }
  % COSH
  \and
  \inferrule{
    f^I [a, b] \subseteq [a', b'] \and
    \widehat{\cosh} [a', b'] = [a'', b'']
  }
  {
    (\cosh f)^I [a, b] \subseteq
    [a'', b'']
  }
  % TANH
  \and
  \inferrule{
    f^I [a, b] \subseteq [a', b'] \and
    \widehat{\tanh} [a', b'] = [a'', b'']
  }
  {
    (\tanh f)^I [a, b] \subseteq
    [a'', b'']
  }
\end{mathpar}


\section{Proof Checking Algorithms}

write about the overall flow, the loop between running dReal and checking. 

\section{Case Study: Kepler Conjecture Benchmarks}




\bibliographystyle{abbrv}
\bibliography{tau}
\end{document}

