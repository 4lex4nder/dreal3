\documentclass[envcountsect]{llncs}
%\usepackage{stmaryrd,amsmath,amssymb,newlfont,graphicx,caption,verbatim}
\usepackage{amsmath,amssymb,newlfont,graphicx,caption,verbatim}
\usepackage[ruled,lined,boxed,commentsnumbered,linesnumbered]{algorithm2e}
\newcommand{\dom}{\mathrm{dom}}
\newtheorem{notation}[theorem]{Notation}
\newcommand{\len}{\mathit{len}}
\newcommand{\poly}{\mathsf{poly}}

%\setlength{\textwidth}{5.7in}
%\setlength{\textheight}{8.2in}
%\setlength{\topmargin}{0in}
%\setlength{\oddsidemargin}{.4in}
%\setlength{\evensidemargin}{.4in}

\title{The {\tt dReal} Tool}
\author{}
\institute{Carnegie Mellon University, Pittsburgh, PA 15213}

\begin{document}
\maketitle

\begin{abstract}
\end{abstract}

\section{Introduction}

\section{$\delta$-Complete Decision Procedures}

\section{Interval Constraint Propagation}

\section{The DPLL(T) Framework}

\section{Input Format and Usage}



\bibliographystyle{abbrv}
\bibliography{tau}
\end{document}

