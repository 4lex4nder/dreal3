\documentclass[envcountsect]{llncs}
%\usepackage{stmaryrd,amsmath,amssymb,newlfont,graphicx,caption,verbatim}
\usepackage{amsmath,amssymb,newlfont,graphicx,caption,verbatim}
\usepackage[ruled,lined,boxed,commentsnumbered,linesnumbered]{algorithm2e}
\newcommand{\dom}{\mathrm{dom}}
\newtheorem{notation}[theorem]{Notation}
\newcommand{\len}{\mathit{len}}
\newcommand{\poly}{\mathsf{poly}}
\newcommand{\dreach}{{\tt dReach}}
%\setlength{\textwidth}{5.7in}
%\setlength{\textheight}{8.2in}
%\setlength{\topmargin}{0in}
%\setlength{\oddsidemargin}{.4in}
%\setlength{\evensidemargin}{.4in}

\title{The {\tt dReach} Tool}
\author{Sicun Gao \and Soonho Kong \and Edmund M. Clarke}
\institute{Carnegie Mellon University, Pittsburgh, PA 15213}

\begin{document}
\maketitle

\begin{abstract}
We present our tool {\tt dReach}, which performs bounded model checking
and invariant validation on nonlinear hybrid systems. The tool provides
correctness guarantees under the framework of $\delta$-complete reachability
analysis. For instance, for bounded model checking, if {\tt dReach} returns that
a system is ``safe'' within some bound, then the system guaranteed to be same
within the bound; on the other hand, if it returns ``$\delta$-unsafe'' (with a
counterexample trace), then there exists some $\delta$-perturbation on the
system that would render it unsafe. Here, $\delta$ is a small numerical error
bound chosen by the user. {\tt dReach} is able to handle various challenging
nonlinear benchmarks. 
\end{abstract}

\section{Introduction}

\section{Tool Design}

\subsection{Unwinding Hybrid Automata}

\subsection{Encoding Invariant Conditions}

\subsection{The Background SMT Solver}

\section{Input Format and Parameters}


\section{Examples and Result}


\bibliographystyle{abbrv}
\bibliography{tau}
\end{document}

