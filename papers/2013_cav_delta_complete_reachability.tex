\documentclass[envcountsect]{llncs}

\usepackage{stmaryrd,amsmath,amssymb,newlfont,graphicx,caption,verbatim}
\usepackage[ruled,lined,boxed,commentsnumbered,linesnumbered]{algorithm2e}

\newtheorem{notation}[theorem]{Notation}

\newcommand{\dom}{\mathrm{dom}}
\newcommand{\len}{\mathit{len}}
\newcommand{\poly}{\mathsf{poly}}
\newcommand{\flow}{\mathsf{flow}}
\newcommand{\jump}{\mathsf{jump}}
\newcommand{\inv}{\mathsf{inv}}
\newcommand{\init}{\mathsf{init}}
\newcommand{\guard}{\mathsf{guard}}
\newcommand{\reset}{\mathsf{reset}}
\newcommand{\reach}{\mathsf{Reach}}
\newcommand{\unsafe}{\mathsf{unsafe}}
\newcommand{\safe}{\mathsf{safe}}
\newcommand{\p}{\mathsf{P}}
\newcommand{\np}{\mathsf{NP}}
\newcommand{\R}{\mathbb{R}}
\newcommand{\lrf}{\mathcal{L}_{\mathbb{R}_{\mathcal{F}}}}

\setlength{\textwidth}{5.3in}
%\setlength{\textheight}{8.2in}
%\setlength{\topmargin}{0in}
\setlength{\oddsidemargin}{.6in}
\setlength{\evensidemargin}{.6in}

\title{$\delta$-Complete Reachability Analysis for Hybrid Systems}
\author{Sicun Gao \and Soonho Kong \and Edmund M. Clarke}
\institute{Carnegie Mellon University, Pittsburgh, PA 15213}

\begin{document}
\maketitle

\begin{abstract}
We propose the notion of ``$\delta$-complete reachability analysis'' for a wide range of hybrid systems. Essentially, a method is called $\delta$-complete if it provides answers with possible but only one-sided  and ``$\delta$-bounded'' errors. Such methods can be applied to any hybrid system that is ``$\lrf$-representable'' (definable by first-order formulas over the reals with an arbitrary set of computable real functions $\mathcal{F}$), following the availability of $\delta$-complete decision procedures. We show how to use bounded model checking and invariant validation techniques to perform $\delta$-complete analysis, and study the theoretical properties of these techniques in this context. We then show promising experimental results of $\delta$-complete techniques on hybrid systems beyond the reach of existing tools. 
\end{abstract}

\section{Introduction}

Hybrid systems with constant dynamics, linear dynamics, differential equations described by polynomials are important classes that have received much attention. However, the generality of the theory of hybrid systems has been restrictive. Exisintg techniques are mostly specific to the systems at hand, and can not go beyond those classes. 

It is well known that such lack of general results come from the evident intractability of the verification problems of hybrid systems. The undeccidability results of simple linear hybrid systems seem to suggest that most of the problems in the area can only be handled in a heuristic manner. 

In this paper we propose a new perspective. 

Our goal is two fold. First, we give a new way of characterizing hybrid systems. We provide a language that is broader than the existing ones. 

It is standard to discuss problems about hybrid systems using the terminology of rational descriptions. This is not in the same line as the normal representation of 

Using first-order logic to describe verification problems of hybrid systems has become standard in state-of-the-art sofware and hardware verification practices. However, the use of first-order logic formulations in the domain of hybrid systems is not widespread. This is largely affected by the difficulty of logic-based methods in handling the real numbers. 

Our results are built on our recent results on the so-called $\delta$-decidability of the theory over the reals with computable functions. It is a general result confirming the decidability of a large class of logic formulas over the real numbers, as long as numerical errors are taken into account in the decision procedures. Such errors have been the topic in hybrid system literature as well. In the work of~\cite{}, it has been argued that admitting errors can save much of the decidability results. However, on the other hand, a different definition of errors has been shown not effective in eliminating the undecidability results. 

  


\section{Computable Functions over the Reals and $\delta$-Complete Decision Procedures}

\subsection{Computable Real Functions}

A computability theory of real functions can be developed with machines that use infinite tapes to encode the real numbers. This approach has been well-developed in {\em Computable Analysis}~\cite{CAbook,Kobook,vasco}, following the initial conception by Turing~\cite{}.

First, to perform any computation over the reals we need a way of representing the real numbers. It is clear that no finite representation is possible. However, any real number can be encoded by an infinite sequence of rational numbers converging to it. 
\begin{definition}[Names]
A name of a real number $a\in \mathbb{R}$ is a function $\mathcal{\gamma}_a: \mathbb{N}\rightarrow \mathbb{Q}$ satisfying that for all $i\in \mathbb{N}$, $|\gamma_a(i) - a|<2^{-i}.$ For $\vec a\in \mathbb{R}^n$, $\gamma_{\vec a}(i) = \langle \gamma_{a_1}(i), ..., \gamma_{a_n}(i)\rangle$. For $\vec a\in \mathbb{R}^n$, we write $\Gamma(\vec a) = \{\gamma: \gamma\mbox{ is a name of }\vec a\}$. 
\end{definition}

Next, we define a real function $f$ to be computable if there is an oracle Turing machine that can take any argument $x$ of $f$ (an infinite tape) as an oracle, and output an infinite sequence of rational approximations to $f(x)$. 

\begin{definition}[Computable Real Functions] We say $f:\subseteq\mathbb{R}^n\rightarrow \mathbb{R}$ is computable, if there exists an oracle Turing machine $\mathcal{M}_f$ that performs the following computation: Let $\vec a\in \dom(f)$ be any argument of $f$ and $\gamma(\vec a)$ any name of $\vec a$. On any input $i\in \mathbb{N}$, $\mathcal{M}_f^{\gamma(\vec a)}(i)$ uses $\gamma(\vec a)$ as an oracle, and computes a $2^{-i}$-approximation to $f(\vec a)$. 
\end{definition}



In the definition, $i$ specifies the desired error bound on the output of $M_f$ with respect to $f(\vec x)$. For any $\vec x\in \dom(f)$, $M_f$ has access to an oracle encoding the name $\gamma_{\vec x}$ of $\vec x$, and output a $2^{-i}$-approximation of $f(\vec x)$. In other words, the sequence $M_f^{\gamma_{\vec x}}(1), M_f^{\gamma_{\vec x}}(2), ...$ is a name of $f(\vec x)$. Intuitively, $f$ is computable if an arbitrarily good approximation of $f(\vec x)$ can be obtained using any good enough approximation to any $\vec x\in\dom(f)$. A key property of this notion of computability is that computable functions over reals are continuous~\cite{CAbook}. Moreover, over any compact set $D\subseteq \mathbb{R}^n$, computable functions are uniformly continuous with a {\em computable modulus of continuity} defined as follows. 

Most common continuous real functions are computable, such as addition, multiplication, absolute value, $\min$, $\max$, $\exp$, $\sin$ and solutions of Lipschitz-continuous ordinary differential equations.~\cite{CAbook} Compositions of computable functions are computable.


Moreover, complexity of real functions can be defined over compact domains. 

\begin{definition}[\cite{Ko92}] Let $D\subseteq \mathbb{R}^n$ be compact. A real function $f:D\rightarrow\mathbb{R}$ is $\mathsf{P}$-computable ($\mathsf{PSPACE}$-computable), if it is computable by an oracle Turing machine $M_{f}^{\gamma(\vec x)}(i)$ that halts in polynomial-time (polynomial-space) for every $i\in \mathbb{N}$ and every $\vec x\in \dom(f)$. 
\end{definition}



We say $f$ is in Type 2 complexity class $\mathsf{C}$ if it is $\mathsf{C}$-computable. $f$ is $\mathsf{C}$-complete if it is $\mathsf{C}$-computable and $\mathsf{C}$-hard~\cite{Kobook}. If $f:D\rightarrow \mathbb{R}$ is $\mathsf{C}$-computable, then it has a $\mathsf{C}$-computable modulus of continuity over $D$. Polynomials, $\exp$, and $\sin$ are all $\mathsf{P}$-computable functions. A recent result~\cite{Kawamura09} established that the complexity of computing solutions of Lipschitz-continuous ODEs over compact domains is a {\sf PSPACE}-complete problem. 

\subsection{$\delta$-Decidability}

We consider first-order formulas with Type 2 computable functions interpreted over the reals. We write $\mathcal{F}$ to denote an arbitrary collection of symbols representing Type 2 computable functions over $\mathbb{R}^n$ for various $n$. We always assume that $\mathcal{F}$ contains at least the constant $0$, unary negation, addition, and the absolute value. (Constants are seen as constant functions.) Let $\mathcal{L_{\mathcal{F}}}$ be the signature $\langle \mathcal{F}, >\rangle$. $\mathcal{L}_{\mathcal{F}}$-formulas are always evaluated in the standard way over the corresponding structure $\mathbb{R}_{\mathcal{F}}= \langle \mathbb{R}, \mathcal{F}, >\rangle$.  

It is not hard to see that we only need to use atomic formulas of the form $t(x_1,...,x_n)>0$ or $t(x_1,...,x_n)\geq 0$, where $t(x_1,...,x_n)$ are built up from functions in $\mathcal{F}$. This follows from the fact that $t(\vec x)=0$ can be written as $-|t(\vec x)|\geq 0$, $t(\vec x)<0$ as $-t(\vec x)>0$, and $t(\vec x)\leq 0$ as $-t(\vec x)\geq 0$. We can then take expressions $s <t $ and $s \leq t$ to abbreviate $t - s > 0$ and $t - s \geq 0$, respectively. Moreover, when a formula is in negation normal form, the negations in front of atomic formulas can be eliminated by replacing $\neg t(\vec x) > 0$ with $-t(\vec x)\geq 0$, and $\neg t(\vec x)\geq 0$ with $-t(\vec x)>0$. In summary, to avoid extra preprocessing of formulas, we give an explicit definition of $\mathcal{L}_{\mathcal{F}}$-formulas as follows.

\begin{definition}[$\mathcal{L}_{\mathcal{F}}$-Formulas]
Let $\mathcal{F}$ be a collection of Type 2 functions, which contains at least $0$, unary negation -, addition $+$, and absolute value $|\cdot|$. We define:
\begin{align*}
t& := x \; | \; f(t(\vec x)), \mbox{ where }f\in \mathcal{F}\mbox{, possibly constant};\\
\varphi& := t(\vec x)> 0 \; | \; t(\vec x)\geq 0 \; | \; \varphi\wedge\varphi \; | \; \varphi\vee\varphi \; | \; \exists x_i\varphi \; |\; \forall x_i\varphi.
\end{align*}
In this setting $\neg\varphi$ is regarded as an inductively defined operation which replaces atomic formulas $t>0$ with $-t\geq 0$, atomic formulas $t\geq 0$ with $-t>0$, switches $\wedge$ and $\vee$, and switches $\forall$ and $\exists$. Implication $\varphi_1\rightarrow\varphi_2$ is defined as $\neg\varphi_1\vee\varphi_2$.
\end{definition}

For notational convenience, from now on we assume that $\mathcal{F}$ always contains all rational constants. 

\begin{definition}[Bounded Quantifiers]
We use the notation of {\em bounded quantifiers}, defined as
\begin{align*}
\exists^{[u,v]}x.\varphi &=_{df}\exists x. ( u \leq x \land x \leq v \wedge \varphi),\\
\forall^{[u,v]}x.\varphi &=_{df} \forall x. ( (u \leq x \land x \leq v) \rightarrow \varphi),
\end{align*}
where $u$ and $v$ denote $\mathcal{L}_{\mathcal{F}}$ terms whose variables only contain free variables in $\varphi$, excluding $x$. It is easy to check that $\exists^{[u,v]}x. \varphi \leftrightarrow \neg \forall^{[u,v]}x. \neg\varphi$. 
\end{definition}

We say a sentence is bounded if it only involves bounded quantifiers. 

\begin{definition}[Bounded $\mathcal{L}_{\mathcal{F}}$-Sentences]
A {\em bounded $\mathcal{L}_{\mathcal{F}}$-sentence} is of the form
$$Q_1^{[u_1,v_1]}x_1\cdots Q_n^{[u_n,v_n]}x_n. \psi(x_1,...,x_n)$$
where $Q_i^{[u_i,v_i]}$s are bounded quantifiers, and $\psi(x_1,...,x_n)$ is a quantifier-free $\mathcal{L}_{\mathcal{F}}$-formula (the matrix). We sometimes write a bounded sentence as $\vec Q^{[\vec u,\vec v]}\vec x.\psi(\vec x)$. 
 
\end{definition}

\begin{remark} Note that by the definition of bounded quantifier, in the bound $[u_1,v_1]$ on the first quantifier, the terms $u_1$ and $v_1$ can only be built from constants in $\mathcal{F}$ since there is no other free variables in 
$$Q_2^{[u_2,v_2]}x_2\cdots Q_n^{[u_n,v_n]}x_n.\psi(x_1,...,x_n),$$ excluding $x_1$. 
\end{remark}


\begin{notation}
We will often write a matrix $\psi(x_1,...,x_n)$ as 
$$\psi[t_1(\vec x)>0,...,t_k(\vec x)>0; t_{k+1}(\vec x)\geq 0,...,t_m(\vec x)\geq 0]$$
to emphasize the fact that $\psi(\vec x)$ is a positive Boolean combination of the atomic formulas shown. 
\end{notation}

We use the conventional notations for the alternation hierarchy. Namely, $\Sigma_n$ (resp. $\Pi_n$) denotes the set of all $\mathcal{L}_{\mathcal{F}}$-sentences in prenex form with $n$ alternating {\em quantifier blocks} starting with $\exists$ (resp. $\forall$). 


We will write a bound $[u,v]$ as $I$ for short. 

\begin{definition}[$\delta$-Variants]
Let $\delta\in \mathbb{Q}^+\cup\{0\}$, and $\varphi$ a bounded $\mathcal{L}_{\mathcal{F}}$-sentence of the form
$$Q_1^{I_1}x_1\cdots Q_n^{I_n}x_n.\psi[t_i>0; t_j\geq 0],$$
where $i\in\{1,...k\}$ and $j\in\{k+1,...,j\}$. The {\em $\delta$-strengthening} $\varphi^{+\delta}$ of $\varphi$ is defined to be the result of replacing each atomic formula $t_i > 0$ by $t_i > \delta$ and each atomic formula $t_j \geq 0$ by $t_j \geq \delta$, that is,
$$Q_1^{I_1}x_1\cdots Q_n^{I_n}x_n.\psi[t_i>\delta; t_j\geq \delta],$$
where $i\in\{1,...k\}$ and $j\in\{k+1,...,j\}$.
Similarly, the {\em $\delta$-weakening} $\varphi^{-\delta}$ of $\varphi$ is defined to be the result of replacing each atomic formula $t_i > 0$ by $t_i > -\delta$ and each atomic formula $t_j \geq 0$ by $t_j \geq -\delta$, that is,
$$Q_1^{I_1}x_1\cdots Q_n^{I_n}x_n.\psi[t_i>-\delta; t_j\geq -\delta].$$
\end{definition}

Note that in the definition, the bounds on the quantifiers are not changed. In fact, we can talk about $\delta$-variants of unbounded formulas as well, which will be mentioned in Section~\ref{negative}. Note also that $\varphi^{+0}$ and $\varphi^{-0}$ are both equivalent to $\varphi$, and that the notions of strengthening and weakening could have been given a uniform definition by allowing $\delta$ to range over positive and negative numbers. We find it a useful mnemonic, however, to have $\varphi^{+\delta}$ denote a slight strengthening of $\varphi$ (the modified atomic constraints make it slightly harder for $\varphi^{+\delta}$ to be true), and to have $\varphi^{-\delta}$ denote a slight weakening.

\begin{theorem}\label{main}
There is an algorithm which, given any bounded $\mathcal{L}_{\mathcal{F}}$-sentence $\varphi$ and $\delta\in \mathbb{Q}^+$, correctly returns one of the following two answers:
\begin{itemize}
\item ``$\mathsf{True}$'': $\varphi$ is true. 
\item ``$\delta$-$\mathsf{False}$": $\varphi^{+\delta}$ is false. 
\end{itemize}
\end{theorem}

Note that the two cases can overlap. If $\varphi$ is true and $\varphi^{+\delta}$ is false, then the algorithm is allowed to return either one. 

\begin{corollary}
There is an algorithm which, given any bounded $\varphi$ and $\delta\in \mathbb{Q}^+$, correctly returns one of the following two answers:
\begin{itemize}
\item ``$\delta$-$\mathsf{True}$'': $\varphi^{-\delta}$ is true. 
\item ``$\mathsf{False}$'': $\varphi$ is false. 
\end{itemize} 
\end{corollary}

\begin{proof}
Apply the previous algorithm to $\neg\varphi$. Proposition~\ref{trivial}, we have $\neg(\varphi)^{+\delta}\leftrightarrow (\neg\varphi)^{-\delta}$. So if $\neg\varphi$ is $\mathsf{True}$ we can report that $\varphi$ is $\mathsf{False}$, and if $\neg\varphi$ is $\delta$-$\mathsf{False}$ we can report that $\varphi$ is $\delta$-$\mathsf{True}$.
\end{proof}

\begin{theorem}\label{compmain}
Let $\mathcal{F}$ be a class of computable functions. Let $S$ be a class of $\mathcal{L}_{\mathcal{F}}$-sentences, such that for any $\varphi$ in $S$, the terms in $\varphi_{[0,1]}$ are computable in complexity class $\mathsf{C}$ where $\mathsf{P_{C[0,1]}\subseteq \mathsf{C}\subseteq \mathsf{PSPACE_{C[0,1]}}}$. Then, for any $\delta\in \mathbb{Q}^+$, the $\delta$-decision problem for bounded $\Sigma_n$-sentences in $S$ is in $\mathsf{(\Sigma_n^P)^C}$.
\end{theorem}


\section{Hybrid Systems, $\lrf$-Representations, and Reachability}

\subsection{Basic Definitions}

A hybrid automaton is an infinite-state machine whose state has a discrete part, ranging over the vertices of a graph, and a continuous part, ranging over some Euclidean space. 
\begin{definition}[Hybrid Automata]\label{auto-def}
An $n$-dimensional hybrid automaton is a tuple $$\langle X, Q, \flow, \guard, \reset, \inv, \init\rangle,$$ where:
\begin{itemize}
\item $X\subseteq \mathbb{R}^n$ specifies the range of the {\bf continuous variables} $\vec x$ of the system. 
\item $Q=\{q_0,...,q_m\}$ is a finite set of discrete {\bf control modes}. 
\item $\flow \subseteq Q\times X\times \R \times X$ is a predicate that specifies the {\bf continuous dynamics} for each mode. Namely, 
\begin{quote}
$\flow = \{(q, \vec a_0, t, \vec a_t):$ In mode $q$, there is a continuous flow from the initial value $\vec x = \vec a_0$ to $\vec x = \vec a_t$, after time $t$.\}
\end{quote}
It is usually given either as explicit mappings from $\vec x_0$ and $t$ to $\vec x_t$, or as solutions of systems of differential equations/inclusions that specify the derivative of $\vec x$ over time, with initial conditions given by $\vec x_0$. 

\item $\jump\subseteq Q\times X\times Q\times X$ is a predicate that specifies the {\bf jump conditions} between two modes. That is,  
\begin{quote}
$\jump$ = \{$(q,q',\vec a,\vec a'):$ In mode $q$, if the continuous variables are of value $\vec x = \vec a$, the automaton may switch to mode $q'$ and change the continuous variables to value $\vec x = \vec a'$.\}
\end{quote}

\item $\inv \subseteq Q\times X$ specifies the {\bf invariant conditions} for the system to stay in a location. Namely,
\begin{quote}
$\inv$ = \{$(q,\vec a)$: When the automaton is in mode $q$, $\vec x = \vec a$ is a possible value for its continuous variables.\}
\end{quote}

\item $\init \subseteq Q\times X$ specifies the set of {\bf initial configurations} of the system. 
\begin{quote}
$\init$ = \{$(q, \vec a)$: The initial configuration of the automaton can be $\vec x = \vec a$ in mode $q$.\}
\end{quote}
It is easy to see that we can require that the system always starts from a single mode. (When there are multiple initial modes, an additional mode can be added to act as the only initial mode.) Thus, without loss of generality we always require that $q=q_0$ in $\init$. 
\end{itemize}
\end{definition}

%\begin{remark}[Jumps]
%The guard conditions and reset functions, put together, define what is often called the {\bf jump conditions}. In fact, sometimes we write 
%$$H = \langle X, Q, \flow, \jump, \inv, \init\rangle,$$
%implicitly assuming that $\jump$ contains the definitions for both $\guard$ and $\reset$. We will give the precise definitions of the jump conditions using logical representations of hybrid automata, in Definition~\ref{lrf-definition}. 
%\end{remark}

\begin{remark}[$\guard$ and $\reset$]
In the $\jump$ conditions, the first two elements $(q, \vec a)$ are sometimes called the $\guard$ conditions, and the other two elements $(q, \vec a')$ are called the $\reset$ conditions. 
\end{remark}

\begin{remark}[Nondeterminism]
According to the definition, a hybrid automaton can be nondeterministic in various ways. For instance: 
\begin{itemize}
\item The $\flow$ and $\jump$ conditions are not restricted to single valued functions. Thus, multiple continuous flows are allowed in each mode, and different discrete jumps are allowed frim each state. 
\item The $\jump$ conditions only specify that a system {\em may} make a transition when the conditions are met. 
\end{itemize}
\end{remark}

\begin{remark}[$\jump$ vs $\inv$] The jump conditions specify when an automaton {\em may} switch to another mode. The invariants (when violated) specify when an automaton {\em must} switch to another mode. In principle, all may-jumps can be reduced to nondeterministic must-jumps. However, it is better to separate the two, and we will later see that they give rise to different logical structures. 
\end{remark}

\begin{definition}[Invariant-Free Hybrid Automata]\index{Invariant-free Hybrid Automata}
We say a hybrid automata $H = \langle X, Q, \flow, \jump, \inv, \init\rangle$ is {\bf invariant-free} if $$\inv = Q\times X.$$ 
We say its invariant is {\em trivial} in this case.  
\end{definition}

\subsection{$\lrf$-Representations}

We now define logical representations of hybrid automata, so that we can express properties of hybrid automata with first-order formulas. 

\begin{definition}[$\lrf$-Representations]\label{lrf-definition}
\index{$\lrf$-Representation}
Let $H$ be an $n$-dimensional hybrid automaton, given as in Definition \ref{auto-def}. Let $\mathcal{F}$ be a set of real functions, and $\mathcal{L}_{\mathbb{R}_{\mathcal{F}}}$ the corresponding first-order language. 

We say that $H$ has an $\lrf$-representation if for every $q,q'\in Q$, there exists quantifier-free $\lrf$-formulas $\phi^q_{\flow}(\vec x, \vec x_0, t)$, $\phi^{q\rightarrow q'}_{\jump}(\vec x, \vec x')$, $\phi^{q}_{\inv}(\vec x)$, $\phi^q_{\init}(\vec x)$ such that for all $\vec a ,\vec a'\in \mathbb{R}^n$, $t\in\mathbb{R}$:
\begin{itemize}
\item $\mathbb{R}\models \phi^q_{\flow}(\vec a, \vec a_0, t)$ iff $(q, \vec a, \vec a_0, t)\in \flow$. 
\item $\mathbb{R}\models \phi^{q\rightarrow q'}_{\jump}(\vec a, \vec a')$, iff, $(q, q', \vec a, \vec a')\in \jump$.
\item $\mathbb{R}\models \phi^q_{\inv}(\vec a)$, iff, $(q, \vec a)\in \inv.$
\item $\mathbb{R}\models \phi^q_{\init}(\vec a)$, iff, $q = q_0$ and $\vec a\in \init_{q_0}$.  
\end{itemize}
We can write $H = \langle X, Q, \phi_{\flow}, \phi_{\jump}, \phi_{\inv}, \phi_{\init}\rangle$ to emphasize that $H$ is represented in this way. But when the context is clear, we simply write $\flow, \jump, \inv, \init$ to denote their logical representations. 
\end{definition}

\begin{remark}
We have not restricted the form of the formulas for defining hybrid automata. This makes the definition more general than necessary. For instance, the flow should be a continuous mapping from $\vec x_0$ and $t$ to $\vec x_t$ (and thus a conjunction of equations of the form $\vec x_t = f(\vec x_0, t)$), instead of arbitrary formulas. Different classes of hybrid systems can be defined by refining this definition.  
\end{remark}


\begin{definition}[Computable Representation]\index{Computable Representation}
%the goal is to prove that computable hybrid automata define computable functions in the skorokhod sense. and then delta-reachability is decidable just because you are deciding computable functions. 
We say a hybrid automaton $H$ has a {\bf computable representation}, if $H$ has an $\lrf$-representation, and all functions in $\mathcal{F}$ are Type 2 computable. 
\end{definition}

From now on we will only consider hybrid automata that have computable representations. 

\subsection{Hybrid Time Domains and Trajectories}

We now specify how the descriptions of hybrid systems are interpreted: They define {\em hybrid trajectories} in the Euclidean space. 

Hybrid automata model systems that exhibit both continuous and discrete behaviors. Consequently the trajectories of hybrid automata are {\em piecewise continuous}. This motivates a two-dimensional structure of time, so that we can keep track of both the discrete changes and the duration of each continuous flow.  

\begin{definition}[Hybrid time structure~\cite{Davoren09}]
The {\bf\em hybrid time structure} $\mathbb{H}$ is a partially-ordered Abelian group $\mathbb{Z}\times\mathbb{R}$ with pairwise addition and group identity $(0,0)$, partially-ordered by $$\leq =_{df} \{ ((i,t),(i',t')) : i\leq i' \wedge t\leq t'\}\subseteq \mathbb{Z}\times\mathbb{R}.$$
 $\mathbb{H}$ is equipped with the norm: $$\|(i,t)\| = \max\{|i|,|t|\},$$
where $|\cdot|$ denotes absolute values. The {\bf\em hybrid future time structure} $\mathbb{H}^+=\mathbb{N}\times \mathbb{R}^+$ is the positive cone of $\mathbb{H}$ equipped with the same norm.
\end{definition}

\begin{remark}
Naturally, $\|\cdot\|$ induces the norm topology $\mathcal{T}_{norm}$ on $\mathbb{H}$ with a basis consisting of $B_{\delta}(t) = \{t'\in \mathbb{H}: \|t'-t\|<\delta, \delta\in \mathbb{R}^+\}.$
\end{remark}

\begin{definition}[Hybrid time domain]
A {\bf\em hybrid time domain} $T$ is a subset, linearly ordered under $\leq$, of $\mathbb{H}^+$ of the form
$$T=\{(i, t): i<m \mbox{ and } t\in [t_i, t_i']\mbox{ or }[t_i, +\infty)\},$$ where%, \mbox{ or }i=m \mbox{ and }t\in [t_m, t_m]\}$$
\begin{itemize}
 \item $i\in \mathbb{N}$, $m\in \mathbb{N}\cup\{+\infty\}$; 
  \item $\{t_i\}_{i=0}^m$ is an increasing sequence in $\mathbb{R}^+$; 
 \item $t_0= 0$ and $t_i'=t_{i+1}$.
 \end{itemize} 
We write the set of all hybrid time domains in $\mathbb{H}^+$ as $D(\mathbb{H})$.
\end{definition}

\begin{remark}
Note how a two-dimensional time domain avoids ``break points'' on the continuous dimension: At the discrete change points, $(i,t_i')$ and $(i+1, t_{i+1})$ are distinguished at the first coordinate. Thus the time intervals for each $i$ can always include both endpoints. 
\end{remark}


\begin{definition}[Hybrid Trajectory Space]
 Let $X\subseteq\mathbb{R}^n$ be an Euclidean space and let $T$ be a hybrid time domain. A {\bf\em hybrid trajectory} is a continuous function 
$$\xi: T \rightarrow X.$$
We write $\Xi_X$ to denote the set of all possible hybrid trajectories from $D(\mathbb{H}^+)$ to $X$. 
 \end{definition}

Now we can define trajectories of a given hybrid automaton. 

\begin{definition}[Trajectories of a Hybrid Automaton]\label{trajec}
Let $H$ be a hybrid automaton, and $\xi: T\rightarrow X$ a hybrid trajectory, where $T$ is of the form:
$$T = \{(i,t): i< m, m\in \mathbb{N}\cup\{+\infty\}, t\in [t_i, t_i'] \mbox{ or } [t_i, +\infty) \}$$
We say that $\xi$ is {\bf\em a trajectory of $H$} of {\bf\em (discrete) depth} $m$, if there exists a function $\sigma^H_{\xi}: \mathbb{N}\rightarrow Q$ such that:
\begin{itemize}
\item $\sigma^H_{\xi}(0) = q_0$ and $\xi(0,0)\in \init_{q_0}$;
\item For all $(i, t)\in T$, $(\sigma^H_{\xi}(i), \xi(i,t))\in \inv$;
\item For $i=0$,
$$\xi(0,t) = \flow(q_0, \xi(0,0), t)$$ 
\item For $i = k+1$ ($0< k+1<m$)
$$\xi(k+1, t) = \flow( \sigma^H_{\xi}(k+1), \xi(k+1, t_{k+1}), t - t_{k+1})$$
where 
$$(\sigma^H(k), \sigma^H(k+1), \xi(k, t_k'), \xi(k+1,t_{k+1}))\in \jump.$$ 
\end{itemize}
We write $\llbracket H\rrbracket$ to denote the set of all trajectories of $H$. We call $\sigma^H_{\xi}$ the {\bf\em labeling function} between $\xi$ and $H$.
\end{definition}

The intuition behind the definition is straightforward. The labeling function $\sigma_{\xi}^H$ is used to label the mode that the system switches to, corresponding to each discrete transition. In each mode, the system flows continuously following the dynamics defined by $\flow(q, \vec x_0, t)$. Note that as defined in the trajectories, $t$ is counted from the initial time stamp. Consequently, $t-t_k$ is the actual duration in the $k$-th mode. When a switch between  two modes is performed, it is required that $\xi(k+1, t_{k+1})$ is updated from the exit value $\xi(k, t_k')$ in the previous mode, following the jump conditions.

\subsection{Reachability and Safety Verification}

The safety/reachability problem for hybrid automata asks whether a subset of the state space of an automaton can be reached by some trajectory of it. 
\begin{definition}[Reachability]\label{reachability}\index{Reachability}
Let $H$ be an $n$-dimensional hybrid automaton, and $U$ a subset of its state space. We say $U$ is reachable by $H$, if there exists $\xi\in\llbracket H \rrbracket$ with its time domain $T_{\mathcal{\xi}}$ and labeling function $\sigma_{\xi}^H$, such that there exists $(i,t)\in T$ satisfying
$$(\sigma^H_{\xi}(i), \xi(i,t))\in U.$$
\end{definition}

\begin{remark}
Although in the definition $U$ is a subset of the hybrid state space $Q\times X$, sometimes we only ask whether a region of the Euclidean space is reachable. Naturally, that is to ask whether the region is reachable in any discrete mode. 
\end{remark}

\begin{example}
Consider the hybrid automaton modelling a bouncing ball in Example~\ref{bball}. It is not hard to see that if the initial value of $x_1$ (height) is some $h\in \mathbb{R}^+$, then in any trajectory of $H_{BB}$, $x_1> h$ can not be reached. On the other hand, any $x_1\in [0, h]$ can be reached by many trajectories of the system. 
\end{example}

In the seminal work of \cite{DBLP:conf/rex/AlurD91,DBLP:conf/hybrid/AlurCHH92}, it is already shown that the reachability problem for simple classes of hybrid automata is already undecidable. In fact, clear boundaries of the decidable and undecidable classes has been given~\cite{DBLP:journals/jcss/HenzingerKPV98}.

\begin{definition}[Linear Hybrid Automata]
Let $\mathcal{F} = \{+\}\cup \mathbb{Q}$ (the rational numbers are considered as 0-ary functions). We say a hybrid automaton $H$ is a {\bf\em linear hybrid automaton} if it has an $\lrf$-representation. 
\end{definition}

Note that a linear hybrid automaton only allow constant dynamics $dx/dt = a$ (otherwise the $\flow$ functions can not be linear in $t$). Also, only rational coefficients are allowed to appear in the jump and invariant conditions. These requirements make it a very limited class of models. However, the reachability problem is already undecidable for the class. 

\begin{proposition}[Undecidability of Reachability for LHA~\cite{DBLP:conf/hybrid/AlurCHH92}]
The reachability problem for Linear Hybrid Automata is undecidable. 
\end{proposition}

It is standard in the existing results that all the constants used in the description of a hybrid automaton are rational, so that symbolic algorithms can be used. This will not be a necessary restriction in our framework. We allow any computable signature $\mathcal{F}$, and any computable real numbers can be used as constants in the describing hybrid automata. It is worth noting that, however, allowing arbitrary computable reals in the description would render the reachability problem for even more trivial classes of hybrid automata to be undecidable. Indeed, since the equality test $x=a$ is undecidable, the question of whether a flow of $dx/dt = 1$ can reach some point $a$ is undecidable already.

\begin{remark}
The reachability problem for timed automata, defined as linear hybrid automata with dynamics of constant 1, is decidable~\cite{DBLP:conf/rex/AlurD91}. In light of the previous remark, such results require the use of rational constants only. Thus we choose not to further discuss the boundaries between decidable and undecidable classes in a traditional setting. 
\end{remark}

\section{Bounded Model Checking}


Bounded model checking is a well-known technique for verifying a transition system up to some bounded depth~\cite{DBLP:journals/fmsd/ClarkeBRZ01}. In this section we show how it works for hybrid automata. 



\section{Encoding Reachable Sets}

We consider hybrid automata with computable representations. We write $\flow$, $\jump$, $\inv$, and $\init$ to denote the corresponding formulas in the $\lrf$-representations of hybrid automata. 

\begin{remark}
Recall the definition of depth in Definition~\ref{trajec}: We only count the number of discrete changes in the trajectories of a hybrid automaton. 
\end{remark}

It will be clear that when we encode reachability for hybrid automata with nontrivial invariant conditions, the logical formulas are more complex. Thus we first consider automata with no invariant conditions, and then the general case. 


\subsection{Automata without Invariants}

A hybrid automaton evolves continuously following $\flow$ in each mode, and in a transition between two modes, the $\jump$ conditions are followed. We define the logic encodings inductively on the number of discrete transitions as follows. 

\begin{definition}[$\reach_H^n(\vec x)$, Invariant-Free Case]

Let $H$ be an invariant-free hybrid automaton $\langle X, Q, \flow, \jump, \init\rangle$. We define the following formulas to encode the reachable states of any bounded depth.  

\begin{itemize}
\item {\bf The Depth $0$ Case.} First, consider the reachable states of $H$ in the initial mode, before making any discrete jump. In this case, the system evolves according to the flow of the initial state. 

Recall that the initial mode is written as $q_0$, and $\flow_{q_0}(\vec x^t, \vec x, t)$ means that in mode $q_0$, if the system starts the continuous flow with $\vec x$ being the initial state of the continuous variables, then after a time duration of $t$, the continuous variables reach the state $\vec x^t$. 

The formula that encodes all the possible $\vec x_0^t$ that the system can reach in the initial mode is the following:
\[
\reach^0_{H,q_0} (\vec x_0^t):=\ \exists \vec x_0 \exists t_0\ \init_{q_0}(\vec x_0)\wedge \flow_q(\vec x_0^t, \vec x_0, t_0)
\]

For any $q\neq q_0$, $\reach^0_{H,q}$ is $\bot$. 

(We write the number of discrete jumps in the subscripts, such as in $\vec x_k$. The superscript $t$ as in $\vec x^t$ is not a variable: $\vec x^t$ denotes the values of the continuous variables after some continuous flow.)


Note that $\vec x_0^t$ are the only free variables in the formula. 



\item {\bf The Depth $k+1$ Case.} Next, consider the induction case in which the system makes a discrete jump from a previous mode. Assume that $\reach^k_{H,q'}$ are defined for any $q'\in Q$. We have:
\begin{multline*}
\reach^{k+1}_{H,q}(\vec x_{k+1}^t):=  \exists \vec x_k^t\exists \vec x_{k+1}\exists t_{k+1}\ \\
\bigvee_{q'\in Q} (\reach^k_{H,q'} (\vec x_k^t) \wedge \jump_{q'\rightarrow q}(\vec x_k^t, \vec x_{k+1}) \wedge \flow_{q}(\vec x_{k+1}^t, \vec x_{k+1}, t_{k+1}))
\end{multline*}
This encoding is intuitive: $\reach_{H,q'}^k(\vec x_k^t)$ defines the states that $H$ can reach, if after $k$ steps of discrete changes it is in mode $q'$. From there, if $H$ makes a $\jump$ from mode $q'$ to $q$, then the states have the make a discrete change following $\jump_{q'\rightarrow q}(\vec x_k^t, \vec x_{k+1})$. As last, in mode $q'$, any state $\vec x_{k+1}^t$ that $H$ can reach should satisfy the $\flow$ conditions $\flow_q(\vec x_{k+1}^t, \vec x_{k+1}, t)$ in mode $q$.

Note that after each discrete jump, a new time variable $t_k$ is introduced and independent from the previous ones.

\item In all, the set of all reachable states of the system, after making $n$ transitions from $q_0$ is encoded by: 
$$\reach^n_{H}(\vec x) := \bigvee_{q'\in Q} \reach^n_{H,q'}(\vec x).$$


\end{itemize}
\end{definition}

\begin{remark}
In $\reach_{q_0}^n(\vec x)$, the existential quantifiers are nested. It is clear that they can be directly prenexed without changing the structure of the formulas. 
\end{remark}

It is not hard to see that the formula defined in this way correctly encodes reachability, as canbe formally proved as follows. 

\begin{proposition}
Let $H$ be an invariant-free hybrid automaton. Then for any $\vec a\in \mathbb{R}^n$, $\reach^n_q(\vec a)$ iff there exists a depth-$n$ trajectory $\xi\in\llbracket H\rrbracket$ such that $\xi(n, t) = \vec a$ for some $t\in\mathbb{R}$, and $\sigma_{\xi}(n) = q$ . 
\end{proposition}


\begin{proof}
We do induction on $n$.

(i) Case $n=0$.

Suppose $\reach^0_{H,q}(\vec a_0^t, \vec a_0, t_0)$ is true, where $\vec a_0\in \mathbb{R}^n$, $\vec a_0^t\in \mathbb{R}^n$, $t_0\in \mathbb{R}$. Then
$$\init_{q_0}(\vec a_0)\wedge\flow_{q_0}(\vec a_t, \vec a_0, t_0) \wedge \unsafe(\vec a^t).$$
Following Definition~\ref{trajec} and Definition~\ref{lrf-definition}, there exists a trajectory $\xi\in \llbracket H\rrbracket$ such that
$$\xi(0,0) = \vec a_0$$
and 
$$\xi(0,t_0) = \vec a_0^t$$
with 
$$\sigma^H_{\xi}(0) = q_0.$$
Then, since $\unsafe(\vec a^t)$, we know that $H$ reaches $\unsafe$ by Definition~\ref{reachability}.

On the other hand, if there is a $\xi\in \llbracket H\rrbracket$ such that $\xi(0,t)$ is in $\unsafe$ for some $t$. Then let $\vec a_0 = \xi(0,0)$, $\vec a_0^t = \xi(0,t)$, and $t_0 = t$. Following the definitions, they witness that $\reach^0_{H,\unsafe}$ is true. 

(ii) Case $n=k+1$. 
 
Suppose $\reach_{H,q}^{k+1}(\vec a^t)$ is true for some $\vec a^t\in\mathbb{R}^n$. Then there exists $\vec b\in \mathbb{R}^n$, $\vec a\in \mathbb{R}$, $t_a\in \mathbb{R}$, and $q'\in Q$ such that
$$\reach_{H,q'}^k(\vec b) \wedge \jump_{q\rightarrow q'}(\vec b, \vec a)\wedge \flow(\vec a^t, \vec a, t_a).$$ 
By inductive hypothesis, since $\reach_{H,q'}^k(\vec b)$, there exists $\xi\in\llbracket H\rrbracket$ and $t_b\in mathbb{R}$ such that $\xi(k, t_b) = \vec b$. 


Now consider the extension of $\xi$ into $\xi'$ satisfying:
$$(q', q, \vec b, \vec a)\in \llbracket \jump_{q\rightarrow q'}\rrbracket$$
and 
$$(\vec a, \vec a^t, t_a)\in \llbracket \flow_q \rrbracket$$
where $\xi'(k,t_b) = \vec b$ and $\xi'(k+1, t_a) = \vec a^t$. Following Definition~\ref{trajec} and Definition~\ref{lrf-definition}, $\xi'\in \llbracket H\rrbracket$. 

On the other hand, suppose there is a trajectory $\xi\in\llbracket H\rrbracket$ such that $\xi(k+1, t_a) = \vec a$. By Definition~\ref{trajec} and Definition~\ref{lrf-definition}, for some $q'\in Q$ we have
$$\reach^n_{H,q'}(\xi(k,t_b))$$
where $t_b\in\mathbb{R}$ satisfying
$$(q', q, \xi(k,t_b), \xi(k+1,t_b))\in \llbracket \jump_{q'\rightarrow q} \rrbracket$$
and
$$(\xi(k+1,t_a), \xi(k+1,t_b), t_a - t_b)\in \llbracket \flow_q\rrbracket$$
Thus we have 
$$\reach_{H,q'}(\vec b)\wedge \jump(q',q,\vec b, \vec a)\wedge \flow(\vec a, \vec a^t, t_a-t_b),$$
which means that $\reach_{H,q}(\vec a^t)$ is true. 
\end{proof}




\begin{definition}[$\reach^n_{H,\unsafe}$] Let $H$ be an invariant-free hybrid automaton. Let $\unsafe_q(\vec x)$ be an $\lrf$-formula. We define
$$\reach^n_{H,\unsafe}:= \exists^X \vec x \ (\reach^n_{H,q}(\vec x)\wedge \unsafe_q(\vec x))$$
\end{definition}

Finally, starting from $\init(\vec x)$ at location $q_0$, if the unsafe state is specified by $\unsafe(\vec x)$ or $\unsafe_{q'}(\vec x)$ (the later has the end location $q'$ specified), we have the formulas expressing that the system reaches the unsafe state in $n$ transitions:
and
$$\exists \vec x\ (\reach^n_{q_0\rightarrow q}(\vec x)\wedge \unsafe_q(\vec x))$$

\begin{corollary}
Let $H$ be an invariant-free hybrid automaton and $\unsafe$ be a subset of its state space. Then $H$ reaches $\unsafe$ in $n$ steps of discrete jumps, if and only if, $\reach^n_{H}$ is true. 
\end{corollary}

\begin{proof}
$H$ reaches $\unsafe$ in $n$ steps iff there is some $\xi\in \llbracket H\rrbracket$ such that $\xi(n, t)\in \llbracket unsafe\rrbracket$. On the other hand, we have shown that $\xi(n, t) = \vec a$ for some $t$ iff $\reach_H^n(\vec a)$.
\end{proof}


\subsection{Automata with Invariants}\label{auto_inv}

If the invariant conditions of a hybrid automaton is nontrivial, we need to ensure that before making a jump, the system always stays in the invariant of its current mode. This requires the use of $\exists\forall$-sentences in the encoding. 

\begin{remark}
It is important to note that in this case, we need to require that the $\flow_q$ predicates are graphs of single continuous functions of the form $\vec x_t = f_q(\vec x_0, t)$. This is because the $\forall$-quantifiers will quantify over all possible states in $\flow_q$, as will be shown in the following encoding. In that case, if $\flow_q$ is nondeterministic, the encoding would be wrong in requiring that all possible flows must satisfy the invariants, while only the selected path needs to satisfy them. 

It is possible to relax such an requirement by allowing that the nondeterminacy in $\flow_q$ can be parametered, such that we can existentially quantify over the parameters to select a deterministic branch of a $\flow_q$. 
\end{remark}


\begin{definition}[$\reach_H^n(\vec x)$, General Case]

Let $H$ be a hybrid automaton $\langle X, Q, \flow, \jump, \inv, \init\rangle$. We define the following formulas. 

\begin{itemize}
\item {\bf The Depth $0$ Case.} First, consider the reachable states of $H$ in the initial mode, before making any discrete jump. In this case, the system evolves according to the flow of the initial state. 

Compared to the case for invariant-free automaton, we need to check whether before any jumping conditions is met, $H$ stays in the same mode satisfying the invariants. This requires universal quantification over all the possible values for the continuous variables before the jumping point. 

The formula that encodes all the possible $\vec x_0^t$ that the system can reach in the initial mode is the following:
\begin{multline*}
\reach^0_{H,q_0} (\vec x_0^t):=\ \exists^X \vec x_0 \exists^{I_t} t_0\forall^{[0,t_0]} t\forall^X \vec x\\ \Big(\init_{q_0}(\vec x_0)\wedge \flow_q(\vec x_0^t, \vec x_0, t_0)\wedge (\flow_q(\vec x, \vec x_0, t)\rightarrow\inv_{q_0}(\vec x))\Big)
\end{multline*}

For any $q\neq q_0$, $\reach^0_{H,q}$ is $\bot$. 

Note that $\vec x_0^t$ are the only free variables in the formula. 



\item {\bf The Depth $k+1$ Case.} Next, consider the induction case in which the system makes a discrete jump from a previous mode. Assume that $\reach^k_{H,q'}$ are defined for any $q'\in Q$. We have:
\begin{multline*}
\reach^{k+1}_{H,q}(\vec x_{k+1}^t):=  \exists^X \vec x_k^t \exists^X \vec x_{k+1}\exists^{T} t_{k+1}\forall^{[0,t_{k+1}]}t\forall^X \vec x \\
\bigvee_{q'\in Q} (\reach^k_{H,q'} (\vec x_k^t) \wedge \jump_{q'\rightarrow q}(\vec x_k^t, \vec x_{k+1}) \wedge \flow_{q}(\vec x_{k+1}^t, \vec x_{k+1}, t_{k+1})\\
\wedge(\flow_q(\vec x, \vec x_{k+1}, t)\rightarrow \inv_q(\vec x)))
\end{multline*}


\item In all, the set of all reachable states of the system, after making $n$ transitions from $q_0$ is encoded by: 
$$\reach^n_{H}(\vec x) := \bigvee_{q'\in Q} \reach^n_{H,q'}(\vec x).$$


\end{itemize}
\end{definition}


\begin{proposition}\label{with-inv}
Let $H = \langle X, Q, \flow, \jump, \inv, \init \rangle$ be a general hybrid automaton. Then for any $\vec a\in \mathbb{R}^n$, $\reach^n_q(\vec a)$ iff there exists a depth-$n$ trajectory $\xi\in\llbracket H\rrbracket$ such that $\xi(n, t) = \vec a$ for some $t\in\mathbb{R}$, and $\sigma_{\xi}(n) = q$. 
\end{proposition}


\begin{proof}
We do induction on $n$.

(i) Case $n=0$.

Suppose $\reach^0_{H,q}(\vec a_0^t, \vec a_0, t_0)$ is true, where $\vec a_0\in \mathbb{R}^n$, $\vec a_0^t\in \mathbb{R}^n$, $t_0\in \mathbb{R}$. Then
$$\init_{q_0}(\vec a_0)\wedge\flow_{q_0}(\vec a_t, \vec a_0, t_0) \wedge \unsafe(\vec a^t).$$
Following Definition~\ref{trajec} and Definition~\ref{lrf-definition}, there exists a trajectory $\xi\in \llbracket H\rrbracket$ such that
$$\xi(0,0) = \vec a_0$$
and 
$$\xi(0,t_0) = \vec a_0^t$$
with 
$$\sigma^H_{\xi}(0) = q_0.$$
Then, since $\unsafe(\vec a^t)$, we know that $H$ reaches $\unsafe$ by Definition~\ref{reachability}.

On the other hand, if there is a $\xi\in \llbracket H\rrbracket$ such that $\xi(0,t)$ is in $\unsafe$ for some $t$. Then let $\vec a_0 = \xi(0,0)$, $\vec a_0^t = \xi(0,t)$, and $t_0 = t$. Following the definitions, they witness that $\reach^0_{H,\unsafe}$ is true. 

(ii) Case $n=k+1$. 
 
Suppose $\reach_{H,q}^{k+1}(\vec a^t)$ is true for some $\vec a^t\in\mathbb{R}^n$. Then there exists $\vec b\in \mathbb{R}^n$, $\vec a\in \mathbb{R}$, $t_a\in \mathbb{R}$, and $q'\in Q$ such that
$$\reach_{H,q'}^k(\vec b) \wedge \jump_{q\rightarrow q'}(\vec b, \vec a)\wedge \flow(\vec a^t, \vec a, t_a).$$ 
By inductive hypothesis, since $\reach_{H,q'}^k(\vec b)$, there exists $\xi\in\llbracket H\rrbracket$ and $t_b\in mathbb{R}$ such that $\xi(k, t_b) = \vec b$. 


Now consider the extension of $\xi$ into $\xi'$ satisfying:
$$(q', q, \vec b, \vec a)\in \llbracket \jump_{q\rightarrow q'}\rrbracket$$
and 
$$(\vec a, \vec a^t, t_a)\in \llbracket \flow_q \rrbracket$$
where $\xi'(k,t_b) = \vec b$ and $\xi'(k+1, t_a) = \vec a^t$. Following Definition~\ref{trajec} and Definition~\ref{lrf-definition}, $\xi'\in \llbracket H\rrbracket$. 

On the other hand, suppose there is a trajectory $\xi\in\llbracket H\rrbracket$ such that $\xi(k+1, t_a) = \vec a$. By Definition~\ref{trajec} and Definition~\ref{lrf-definition}, for some $q'\in Q$ we have
$$\reach^n_{H,q'}(\xi(k,t_b))$$
where $t_b\in\mathbb{R}$ satisfying
$$(q', q, \xi(k,t_b), \xi(k+1,t_b))\in \llbracket \jump_{q'\rightarrow q} \rrbracket$$
and
$$(\xi(k+1,t_a), \xi(k+1,t_b), t_a - t_b)\in \llbracket \flow_q\rrbracket$$
Thus we have 
$$\reach_{H,q'}(\vec b)\wedge \jump(q',q,\vec b, \vec a)\wedge \flow(\vec a, \vec a^t, t_a-t_b),$$
which means that $\reach_{H,q}(\vec a^t)$ is true. 
\end{proof}











\section{Solving Bounded $\delta$-Reachability}


\subsection{Perturbations on Hybrid Automata}
Perturbations on hybrid automata is defined as follows. 

\begin{definition}[Perturbations on Hybrid Automata]
Suppose $\delta\in\mathbb{Q}^+\cup\{0\}$, and 
$$H = \langle X, Q, \flow, \jump, \inv, \init\rangle$$
is an $\lrf$-represented hybrid automaton. We define the {\bf \em $\delta$-weakening} of $H$ as
$$H^{\delta} = \langle X, Q, \flow^{-\delta}, \jump^{-\delta}, \inv^{-\delta}, \init^{-\delta}\rangle$$
\end{definition}

\begin{remark}
Note that we only defined the weakening of a hybrid automaton here, because weakening essentially corresponds to overapproximations of the reachable states. It is conceivable that for other tasks, strengthening will be needed (for instance for synthesizing parameters). But this will not be our focus for now. 
\end{remark}

Since the predicates in the definition of $H$ are all quantifier free. It is clear that $H^{\delta}$ is an overapproximation of $H$. 
\begin{proposition} Let $\delta\in\mathbb{Q}^+\cup \{0\}$ and $H$ be given. We always have
$$\llbracket H\rrbracket\subseteq \llbracket H^{\delta}\rrbracket.$$
\end{proposition}
\begin{proof}
Let $\xi\in \llbracket H\rrbracket$ be any trajectory of $H$. Recall that for any $\lrf$ sentence $\varphi$, $\varphi\rightarrow\varphi^{-\delta}$. Now since $\xi$ satisfies the conditions in Definition~\ref{trajec}, by replacing each predicate in the definition by their $\delta$-weakenings, we know $\xi\in \llbracket H^{\delta}\rrbracket$.
\end{proof}
\begin{remark}
The $\delta$-weakening of any deterministic hybrid automaton is nondeterministic. 
\end{remark}

\begin{remark}
$H^{\delta}$ approximates $H$ {\em syntactically}. Semantically, it can be tricky to quantify the effect of $\delta$. If we are dealing with a continuous system, it is clear that the the Hausdorff distance between $\llbracket H^{\delta}\rrbracket$ and $\llbracket H\rrbracket$ is a continuous function in $\delta$. However, for hybrid trajectories, relaxing the guards in a system can generate very different trajectory sets. It is worth noting how the syntactic definition avoids such complications. 
\end{remark}
\begin{example}
Let $H$ be defined as:
\begin{eqnarray*} 
X &:& \mathbb{R}\\
Q &:& \{q_0, q_1\}\\
\flow_{q_0} &:& x = 1\\
\flow_{q_1} &:& dx/dt = 1\\
\jump_{q_0\rightarrow q_1} &:& x >1 \wedge x' = x\\
\init_{q_0} &:& x=1
\end{eqnarray*}
Note that the jump condition is not met for any trajectory of $H$. Thus for all $\xi\in \llbracket H\rrbracket$, 
$$\xi(0, t) = 1\mbox{ and }\xi(k,t)\mbox{ is undefined for any $k>0$.}$$
On the other hand, for any $\delta\in \mathbb{Q}^+$, we have that for any $a > 1$, there is some $\xi\in \llbracket H^{\delta}$ and some $t\in \mathbb{R}$ such that
$$\xi(1, t) = a.$$
Consequently for any such $\delta$, the Hausdorff distance between the two trajectory sets is
$$d_{H}(\llbracket H\rrbracket, \llbracket H^{\delta}\rrbracket) = \infty$$
in the generalized Skorokhod topology. 
\end{example}

\begin{proposition}\label{equiv-delta}
$(\reach_{H,q}^n(\vec a))^{-\delta}$ iff there is a trajectory $\xi\in \llbracket H^{\delta}\rrbracket$ such that $\xi(n, t) = \vec a$ and $\sigma_{\xi}(n) = q$.  
\end{proposition}

\subsection{Bounded $\delta$-Reachability}

We focus on the problem of bounded $\delta$-Reachability, defined as follows. 

\begin{definition}[Bounded $\delta$-Reachability]
Let $H$ be a hybrid automaton such that the state space of $H$ is a bounded set. Let $U$ be a subset of $X$. Let $n\in \mathbb{N}$, and $T\in \mathbb{R}$ a bounded interval. The bounded $\delta$-reachability asks for one of the following decisions:
\begin{itemize}
\item $\delta$-$\unsafe$: $H^{\delta}$ reaches $\unsafe^{-\delta}$.  
\item $\safe$: $H$ does not reach $U$ in $n$ steps within time $T$. 
\end{itemize} 
When the two cases overlap, either answer can be returned. 
\end{definition}

\begin{remark}
There are three bounds here. First, the state space $X$ and the time duration $T$ are bounded. These two constraints ensure that the variables that will appear in the logical encoding are bounded. Second, the number of steps is bounded. This allows us to encode the problem with a finite formula. 
\end{remark}


\begin{lemma}\label{bmclemma}
Let $H$ be a hybrid automaton, $\unsafe$ a subset of its state space, and $n\in\mathbb{N}$. We have:
\begin{itemize}
\item If $(\reach^n_{H,\unsafe})^{-\delta}$ is true, then $H^{\delta}$ is unsafe. 
\item If $\reach^n_{H,\unsafe}$ is false, then $H$ is safe up to $n$. 
\end{itemize}
\end{lemma}

\begin{proof}
Following Proposition~\ref{equiv-delta}, if $(\reach^n_{H,\unsafe})^{-\delta}$ is true then there exists $\xi\in \llbracket H^{\delta}\rrbracket$ such that $\xi(0,t)=\vec a$ for some $t$, such that 
$$(\reach^n_{H,q}(\vec a))^{-\delta} \wedge (\unsafe(\vec a))^{-\delta}$$
is true. Thus $H^{\delta}$ is unsafe, witnessed by $\vec a$. 

On the other hand, suppose $\reach^n_{H,\unsafe}$ is false. Following Proposition~\ref{equiv-delta}, there does not exists any $\xi$ such that $\xi(n,t) = \vec a$ and $\unsafe(\vec a)$ is true. Consequently, $H$ is safe up to depth $n$. 
\end{proof}




\begin{theorem}
Let $\mathcal{F}$ be a set of computable functions. There exists an algorithm such that, for any given $\lrf$-representable $H$ and an unsafe region $\unsafe$, and a time bound $T$, decide the bounded $\delta$-reachability problem for $H$.  
\end{theorem}

\begin{proof}
Since $H$ and $\unsafe$ are both $\lrf$-representable, $\reach_{H,unsafe}$ is an $\lrf$-sentence. Following Corollary~\ref{weaken}, there exists an algorithm that correctly decides:
\begin{itemize}
\item $\reach_{H,\unsafe}$ is false, or
\item $(\reach_{H,\unsafe})^{-\delta}$ is true. 
\end{itemize}
Following Lemma~\ref{bmclemma}, the first case implies that $H$ is safe up to depth $n$ within time bound $T$, and the second case implies that $H$ is unsafe. 
\end{proof}

It is clear that all we need is a $\delta$-complete decision procedure to solve the bounded $\delta$-reachability problems. 

\begin{definition}[$\delta$-Complete Bounded Model Checking]
We say a bounded model checking algorithm is $\delta$-complete, if it solves the $\delta$-reachability problem correctly. 
\end{definition}
\begin{algorithm}\caption{$\delta$-Complete Bounded Model Checking}\label{delta-bmc}

\SetKwInOut{Input}{input}\SetKwInOut{Output}{output}
\Input{An $\lrf$-represented hybrid automaton $\varphi = \langle X, Q, \flow, \jump, \inv, \init\rangle$, time bound $T\in \mathbb{R}$, depth $n$}
\Output{Safe or $\delta$-Unsafe}
\BlankLine
$\reach_{H}(0,q_0,\vec x_0') \leftarrow \exists^X \vec x_0\exists^{[0,T]}t_0\Big( \init_{q_0}(\vec x_0)\wedge \flow_{q_0}(\vec x_0', \vec x_0, t_0)\Big)$\\
$k\leftarrow 0$\\
\While{$k<n$}
{
$\reach_{H}(k+1, q, \vec x_{k+1}') \leftarrow \exists^X\vec x_k'\exists^X x_{k+1}\exists^{[0,T]}t_{k+1} \bigvee_{q\in Q}\Big(\reach_{H}(k, q, \vec x_k')\wedge \flow(\vec x_k', \vec x_k, t_k)\wedge \forall^X \vec x\forall^{[0,t_k]} t (\flow(\vec x, \vec x_k', t)\rightarrow \inv(\vec x))\Big)$\\
$k\leftarrow k+1$
}
$\reach_{H,U} \leftarrow \exists \vec x(\reach_{H}(n, q, \vec x)\wedge \unsafe(\vec x))$
\\
\eIf{$\reach_{H,U}$ is {\sf $\delta$-true}}
{$H$ is {\sf $\delta$-unsafe}}
{$H$ is {\sf safe} up to depth $n$ within time bound $T$}
\end{algorithm}
It is clear that Algorithm \ref{delta-bmc} is $\delta$-complete if it uses a $\delta$-complete decision procedure. 

Note that in general, unbounded $\delta$-reachability is still undecidable.
\begin{proposition}
Unbounded-time $\delta$-Reachability is undecidable in a bounded space.  
\end{proposition}

\begin{proof}
(Error bounds for reasoning about chaotic systems can not be determined if time is unbounded. )
\end{proof}



\subsection{Complexity Results}


We have reduced the bounded $\delta$-reachability problems to the decision problems of $\lrf$-sentences. Thus, we can obtain the following complexity results of the reachability problems. 


\begin{theorem}
Suppose all the functions in the description of $H$ is in Type 2 complexity class $\mathsf{C}$. Then deciding the bounded $\delta$-Reach problem of Depth $n$, in time bounded by $T$, for an invariant-free $H$ is in $\np^{\mathsf{C}}$, and the problem for $H$ with invariant is in $\Sigma_2^{\mathsf{C}}$. 
\end{theorem}

\begin{proof}
Note that the formula $\reach_{H,\unsafe}$ is exponential in the size of the description of $H$. However, we can nondeterministically select a sequence of $n$ modes, and check if reachability can be witnessed.

(i) For an invariant-free $H$, the formula encoding one possible sequence of $n$ steps of mode change is of the form:
\begin{eqnarray*}
& &\exists \vec x_0 \exists \vec x_0'\exists t_0\cdots \exists \vec x_{n-1}\exists\vec x_{n-1}'\exists t_{n-1}\\
& &\Big(\init(\vec x_0)\wedge \flow_{q_0}(\vec x_0, \vec x_0', t_0)\wedge \jump_{q_0\rightarrow q_1}(\vec x_0', \vec x_1)\wedge \cdots\\
& &\wedge \flow_{q_{n-1}}(\vec x_{n-1}, \vec x_{n-1}', t_{n-1}) \wedge \unsafe(\vec x_{n-1}')\Big)
\end{eqnarray*}
where $q_1,...,q_{n-1}$ are random modes in $Q$ ($q_0$ is always the initial one). This formula is polynomial in $H$, $n$, $T$. Thus, we can use the nondeterministic machine to randomly first select such a formula in polynomial time, and then $\delta$-decide its truth value, using the $\mathsf{C}$ oracle. In all, the computation can be performed by a nondeterministic machine with a $\mathsf{C}$-oracle in polynomial time. Thus, the $\delta$-reachability problem is in $\mathsf{NP^C}$, following . On the other hand, any of the predicate, for instance $\init(\vec x_0)$, is allowed to be an arbitrary quantifier-free $\lrf$-formula. Thus it can encode any propositional logic formula as needed in Corollary~\ref{Completeness}.

(ii) For a general $H$, the formula encoding one possible sequence of $n$ steps of mode change is of the form:
\begin{eqnarray*}
& &\exists \vec x_0 \exists \vec x_0'\exists t_0\cdots \exists \vec x_{n-1}\exists\vec x_{n-1}'\exists t_{n-1} \forall \vec x \forall^{[0,t_0]}t_0'\cdots \forall^{[0,t_{n-1}]}t_{n-1}'\\
& &\Big(\init(\vec x_0)\wedge \flow_{q_0}(\vec x_0, \vec x_0', t_0)\wedge \jump_{q_0\rightarrow q_1}(\vec x_0', \vec x_1)\\
& &\wedge (\flow_{q_0}(\vec x, \vec x_0, t_0')\rightarrow \inv_{q_0}(\vec x))\wedge \cdots\\
& &\wedge\jump_{q_{n-2}\rightarrow q_{n-1}}(\vec x_{n-2}', \vec x_{n-1}) \wedge \flow_{q_{n-1}}(\vec x_{n-1}, \vec x_{n-1}', t_{n-1})\\
& &\wedge (\flow_{q_{n-1}}(\vec x, \vec x_{n-1}, t_{n-1}')\rightarrow \inv_q(\vec x))\\
& &\wedge \unsafe(\vec x_{n-1}')\Big)
\end{eqnarray*}
where $q_1,...,q_{n-1}$ are random modes in $Q$. Using a nondeterministic Turing machine, such a formula is first selected in polynomial time, and its truth value is $\delta$-decided as a $\Sigma_2$-sentence. Thus similarly as the invariant-free case, it follows from Theorem~\ref{compmain} and Corollary~\ref{Completeness} that the $\delta$-reachability problem is $\mathsf{\Sigma_2^C}$-complete. 

\end{proof}

\begin{remark}
The input parameters of the $\delta$-reachability problem are: the depth $n$, the time bound $T$, the length of the $\lrf$-description of $H$, and the error bound $\delta$. 
\end{remark}





\section{Invariant Validation}

\section{Experiments}

\section{Conclusion}





\bibliographystyle{abbrv}
\bibliography{tau}

\newpage
\section*{Appendix}

\subsection*{Proof of Proposition \ref{equiv-delta}}
\begin{proof}
Prove by induction on $n$.

(i) Case $n=0$.

Suppose $(\reach^0_{H,q}(\vec a_0^t))^{-\delta}$ is true, where $\vec a_0\in \mathbb{R}^n$. Then there exists $\vec a_0^t\in \mathbb{R}^n$ and $t_0\in \mathbb{R}$ such that
$$(\init_{q_0}(\vec a_0))^{-\delta}\wedge(\flow_{q_0}(\vec a_t, \vec a_0, t_0))^{-\delta}.$$
Following Definition~\ref{trajec} and Definition~\ref{lrf-definition}, there exists a trajectory $\xi\in \llbracket H\rrbracket$ such that
$$\xi(0,0) = \vec a_0$$
and 
$$\xi(0,t_0) = \vec a_0^t$$
with 
$$\sigma^H_{\xi}(0) = q_0.$$
Then, since $\unsafe(\vec a^t)$, we know that $H$ reaches $\unsafe$ by Definition~\ref{reachability}.

On the other hand, if there is a $\xi\in \llbracket H\rrbracket$ such that $\xi(0,t)$ is in $\unsafe$ for some $t$. Then let $\vec a_0 = \xi(0,0)$, $\vec a_0^t = \xi(0,t)$, and $t_0 = t$. Following the definitions, they witness that $\reach^0_{H,\unsafe}$ is true. 

(ii) Case $n=k+1$. 
 
Suppose $\reach_{H,q}^{k+1}(\vec a^t)$ is true for some $\vec a^t\in\mathbb{R}^n$. Then there exists $\vec b\in \mathbb{R}^n$, $\vec a\in \mathbb{R}$, $t_a\in \mathbb{R}$, and $q'\in Q$ such that
$$\reach_{H,q'}^k(\vec b) \wedge \jump_{q\rightarrow q'}(\vec b, \vec a)\wedge \flow(\vec a^t, \vec a, t_a).$$ 
By inductive hypothesis, since $\reach_{H,q'}^k(\vec b)$, there exists $\xi\in\llbracket H\rrbracket$ and $t_b\in mathbb{R}$ such that $\xi(k, t_b) = \vec b$. 


Now consider the extension of $\xi$ into $\xi'$ satisfying:
$$(q', q, \vec b, \vec a)\in \llbracket \jump_{q\rightarrow q'}\rrbracket$$
and 
$$(\vec a, \vec a^t, t_a)\in \llbracket \flow_q \rrbracket$$
where $\xi'(k,t_b) = \vec b$ and $\xi'(k+1, t_a) = \vec a^t$. Following Definition~\ref{trajec} and Definition~\ref{lrf-definition}, $\xi'\in \llbracket H\rrbracket$. 

On the other hand, suppose there is a trajectory $\xi\in\llbracket H\rrbracket$ such that $\xi(k+1, t_a) = \vec a$. By Definition~\ref{trajec} and Definition~\ref{lrf-definition}, for some $q'\in Q$ we have
$$\reach^n_{H,q'}(\xi(k,t_b))$$
where $t_b\in\mathbb{R}$ satisfying
$$(q', q, \xi(k,t_b), \xi(k+1,t_b))\in \llbracket \jump_{q'\rightarrow q} \rrbracket$$
and
$$(\xi(k+1,t_a), \xi(k+1,t_b), t_a - t_b)\in \llbracket \flow_q\rrbracket$$
Thus we have 
$$\reach_{H,q'}(\vec b)\wedge \jump(q',q,\vec b, \vec a)\wedge \flow(\vec a, \vec a^t, t_a-t_b),$$
which means that $\reach_{H,q}(\vec a^t)$ is true. 


\end{proof}



\end{document}

